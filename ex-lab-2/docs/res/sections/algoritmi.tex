\section{Descrizione degli algoritmi}

\subsection{Struttura dati per il grafo}

La struttura dati per il grafo è stata implementata nel seguente
modo:

\subsection{Held e Karp}

\subsubsection{Introduzione}


\subsubsection{Implementazione}

\subsubsection{Ottimizzazioni implementate}

\subsection{Nearest Neighbor}

\subsubsection{Introduzione}

\subsubsection{Implementazione}

\subsubsection{Ottimizzazioni implementate}

\subsection{2-approssimato}

\subsubsection{Introduzione}

L'algortimo 2-approssimato è in grado di determinare un'approssimazione del TSP sotto la condizione che le
distanze rispettino la \textit{disuguaglianza triangolare}, ovvero:
\begin{equation}
    \forall u, v, w \in V, \textnormal{ vale che } c(u, v) \le c(u, w) + c(w, v)
\end{equation}

dove $c$ è la \textit{funzione di costo}. Questo implica che il cammino con un lato $c(u, v)$ ha un costo
minore o uguale al costo del cammino con due lati $c(u, w, v)$. Il nome di questo problema è \textbf{TRIANGLE\_TSP}.

\subsubsection{Definizione di MST}

Sia $V$ l'insieme dei nodi che costituiscono il grafo pesato $G$ e sia $E$ la collezione dei lati di tale
grafo. Ai fini delle analisi della complessità degli algoritmi, sia $|V| = n$ e $|E| = m$.

Un \textit{minimum spanning tree} è un
sottoinsieme dei lati $E$ di un grafo $G$ non orientato connesso e pesato sui lati che
collega tutti i vertici insieme, senza alcun ciclo e con il minimo peso totale del
lato possibile. Cioè, è uno spanning tree la cui somma dei pesi dei bordi è la più
piccola possibile.

Un \textit{minimum spanning tree} $T = (V, E')$ è un albero, il cui insieme dei lati $E'$ è un
sottoinsieme dei lati $E$ di un grafo $G = (V, E)$ non orientato, connesso e
pesato, che collega tutti i vertici $V$, la cui somma dei pesi dei lati è la minima.

L'algoritmo generico per determinare un MST è:
\begin{verbatim}
    A = empty_set
    while A doesn't form a spanning tree
        find an edge (u,v) that is safe for A
        A = A U {(u,v)}
    return A
\end{verbatim}

Si forniscono alcune definizioni per gli MST:
\begin{enumerate}
    \item un \textbf{taglio} $(S, V \setminus S)$ di un grafo $G = (V, E)$ è una partizione
    di $V$;
    \item un lato $(u, v) \in E$ \textbf{attraversa il taglio} $(S, V \setminus S)$ se
    $u \in S$ e $v \in V \setminus S$ o viceversa;
    \item un taglio \textbf{rispetta} un insieme $A$ di lati se nessun lato di $A$ attraversa
    il taglio;
    \item dato un taglio, il lato che lo attraversa di peso minimo si chiama \textbf{light edge}.
\end{enumerate}

Per determinare se un lato è \textbf{safe}, si sfrutta il seguente teorema:

\textbf{Teorema}: Sia $G = (V, E)$ un grafo non diretto, connesso e pesato. Sia $A$ un
sottoinsieme di $E$ incluso in una qualche MST di $G$, sia $(S, V \setminus S)$ un
taglio che rispetta $A$, e sia $(u, v)$ un \textit{light edge} per $(S, V \setminus S)$.
Allora $(u, v)$ è \textit{safe} per $A$.

\subsubsection{Algoritmo}

L'idea che sta dietro a questo algoritmo consiste nell'utilizzare un algoritmo per il calcolo
del MST. Tuttavia, il MST è un albero e quello che vogliamo ottenere invece è un
ciclo hamiltoniano. Per fare ciò, eseguiamo una visita in \textit{preorder} del MST
e aggiungiamo la radice di tale MST alla lista della determinata dalla preorder. Questo è un ciclo
hamiltoniano del grafo originale, in quanto quest'ultimo è completo. Pertanto, esiste sempre un
lato tra ogni coppia di vertici.

\begin{verbatim}
    2-APPROXIMATION(G=(V,E), c)
        root <- v1 in V         // scelgo un nodo di V come radice per Prim
        T <- Prim(G, c, root)
        H' <- preorder(root)    // visita in preorder
        H <- H U {root}           // aggiungo il percorso che va dalll'ultimo nodo
                                // della visita in preorder alla radice
        return H

\end{verbatim}

\subsubsection{Analisi della qualità della soluzione}

Si illustri l'analisi della qualità della soluzione ritornata dall'algoritmo:
\begin{enumerate}
    \item Il costo di $H'$ è basso per la definizione di MST;
    \item supponiamo che presi due vertici $a$ e $b$ non siano collegati da un lato (anche se
    sappiamo che il grafo è completo), L'idea è che il costo di $(a, b)$ è minore rispetto
    al costo di fare un giro più largo, in quanto vale la \textit{disuguaglianza triangolare}.
    Quindi in questo caso $(a, b)$ è un \textit{shortcut}.
\end{enumerate}

\subsubsection{Analisi del fattore di approssimazione}
Si illustri l'analisi del fattore di approssimazione dell'algoritmo:
\begin{enumerate}
    \item \textit{Limite inferiore al costo della soluzione ottima $H$}: sia $H$ il ciclo
    ottimo $H^{ottimo} = <v_{j1}, ..., v_{jn}, v_{j1}>$.
    Sia $H'^{ottimo} = <v_{j1}, ..., v_{jn}>$ un cammino
    (è uno spanning tree, non un ciclo) hamiltoniano. Quindi, $c(H') \ge c(T)$, dove $T$ è
    il MST, e $c(H^{ottimo}) \ge c(H'^{ottimo})$ prechè i costi sono maggiori o uguali a zero.

    \item \textit{Limite superiore al costo della soluzione restituita $H$}: dato un albero,
    una \textit{full preorder chain} è una lista con ripetizioni dei nodi dell'albero che
    indica i nodi raggiunti dalle chiamate ricorsive dell'algoritmo \verb|Preorder|.

    \textit{Proprietà}: in una full preorder chain ogni arco di $T$ appare esattamente due
    volte. Quindi $c(\textnormal{full preorder chain}) = 2 \cdot c(T)$. Se si eliminano
    dalla full preorder chain tutte le occorrenze successive alla prima dei nodi interni
    (tranne l'ultima occorrenza della radice) otteniamo, grazie alla
    \textit{disuguaglianza triangolare}:
    \[
        2 \cdot c(T) \ge c(H) \Rightarrow 2 \cdot c(H^{ottimo}) \ge 2 \cdot c(T) \ge c(H)
        \Rightarrow \frac{c(H)}{c(H^{ottimo})} \le 2
    \]

\end{enumerate}

\subsubsection{Implementazione}

\subsubsection{Ottimizzazioni implementate}

