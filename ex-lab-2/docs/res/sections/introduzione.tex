\section{Introduzione}

\subsection{Descrizione del problema}

In questa relazione illustreremo dei confronti tra tre algoritmi per risolvere un problema
intrattabile, confrontando i tempi di calcolo e la \textit{qualità} delle soluzioni
che si possono ottenere con \textbf{algoritmi esatti} e con \textbf{algoritmi di approssimazione}.
Il problema in questione è il \textbf{Travelling Salesman Problem} (\textbf{TSP} o
\textit{Problema del Commesso Viaggiatore}). Il nome di questo problema deriva dalla sua
rappresentazione: data una rete di città, connesse tra loro tramite strade, si determini il
percorso di minore distanza che un commesso viaggiatore deve fare per visitare tutte le città
\textit{una ed una sola volta}.

Il TSP si può rappresentare con un grafo non orientato, pesato e
\textit{completo} $G = (V,E)$, dove i vertici sono le città ed il peso del lato ${u,v}$
è uguale alla distanza da $u$ a $v$. Risolvere il TSP significa trovare un
\textbf{circuito Hamiltoniano}, ovvero, un ciclo di costo minimo che visita tutti
i vertici \textit{esattamente una volta}.

\subsection{NP-completezza del problema}

Precedentemente è stato detto che il fatto di risolvere il TSP corrisponde a risolvere
il problema della ricerca di un circuito Hamiltoniano. Quest'ultimo problema è noto
essere un problema \textit{NP-completo}, pertanto anche TSP sarà NP-completo.
Il fatto di stabilire che TSP è un problema NP-completo, indica che molto probabilmente
non esiste una soluzione polinomiale al problema.

Per poter stabilire con precisione la sua complessità dobbiamo prima
“trasformarlo” in un problema di decisione, aggiungendo un limite k per il peso del ciclo
all’input del problema.

\textbf{Teorema}: se $\mathcal{P} \ne \mathcal{NP}$, non può esistere alcun algoritmo
di $\rho$-approssimazione per TSP con $\rho = \mathcal{O}(1)$.

\textit{Dimostrazione}: se per assurdo si supponga che $\exists$ un algoritmo $A_\rho$
polinomiale di $\rho$-approssimazione per TSP, dimostro come costruire $A_{Hamilton}$
che decide il problema del ciclo Hamiltoniano in tempo polinomiale. Sia $I = G =(V,E)$
e $O = G$ contiene un ciclo Hamiltoniano? Si effettui una \textit{riduzione}:
\[
    G \rightarrow G' = (V, E') \textnormal{ completo}
    c(e \in E') = 1 \textnormal{ se } e \in E \textnormal{, } \rho|V| + 1 \textnormal{ altrimenti.}
\]
Eseguo $A_\rho(G') \rightarrow C \textnormal{ (ciclo), } c(C) \textnormal{(costo di $C$)}$ e si
determina:
\begin{enumerate}
\item $G \in HAMILTON \Rightarrow c(C^*) = |V| \Rightarrow A_\rho$, ritorna un ciclo $C$
con $c(C) \le \rho|V|$;
\item $G \not\in HAMILTON \Rightarrow C$ contiene almeno un lato non in
$G \Rightarrow c(C) \ge \rho|V| + 1$.
\end{enumerate}

È possibile notare che si può costruire il grafo $G'$ in tempo polinomiale rispetto al numero
di vertici $|V|$ del grafo di partenza $G$. Il grafo $G$ contiene un circuito Hamiltoniano se e
solo se $G'$ ha un ciclo di peso minore o uguale a 0. Supponiamo che $G$ contenga un circuito
Hamiltoniano $h = v_1, ..., v_n$. Ogni lato che compone $h$ è presente in $E$ e quindi ha peso 0 in $G'$.
Di conseguenza, $h$ è un ciclo di $G'$ di peso uguale a 0. Viceversa, supponiamo che $G'$ contenga un ciclo
semplice $t$ che attraversa tutti i vertici e di peso minore o uguale a 0. Poiché i pesi dei lati
in $G'$ sono solo 0 oppure 1, tutti i lati che compongono $t$ devono avere costo 0. Quindi tutti i
lati del ciclo sono presenti anche in $E$ e $t$ è un circuito Hamiltoniano per $G$.

\subsection{Definizione di MST}

Sia $V$ l'insieme dei nodi che costituiscono il grafo pesato $G$ e sia $E$ la collezione dei lati di tale
grafo. Ai fini delle analisi della complessità degli algoritmi, sia $|V| = n$ e $|E| = m$.

Un \textit{minimum spanning tree} è un
sottoinsieme dei lati $E$ di un grafo $G$ non orientato connesso e pesato sui lati che
collega tutti i vertici insieme, senza alcun ciclo e con il minimo peso totale del
lato possibile. Cioè, è uno spanning tree la cui somma dei pesi dei bordi è la più
piccola possibile.

Un \textit{minimum spanning tree} $T = (V, E')$ è un albero, il cui insieme dei lati $E'$ è un
sottoinsieme dei lati $E$ di un grafo $G = (V, E)$ non orientato, connesso e
pesato, che collega tutti i vertici $V$, la cui somma dei pesi dei lati è la minima.

L'algoritmo generico per determinare un MST è:
\begin{verbatim}
    A = empty_set
    while A doesn't form a spanning tree
        find an edge (u,v) that is safe for A
        A = A U {(u,v)}
    return A
\end{verbatim}

Si forniscono alcune definizioni per gli MST:
\begin{enumerate}
    \item un \textbf{taglio} $(S, V \setminus S)$ di un grafo $G = (V, E)$ è una partizione
    di $V$;
    \item un lato $(u, v) \in E$ \textbf{attraversa il taglio} $(S, V \setminus S)$ se
    $u \in S$ e $v \in V \setminus S$ o viceversa;
    \item un taglio \textbf{rispetta} un insieme $A$ di lati se nessun lato di $A$ attraversa
    il taglio;
    \item dato un taglio, il lato che lo attraversa di peso minimo si chiama \textbf{light edge}.
\end{enumerate}

Per determinare se un lato è \textbf{safe}, si sfrutta il seguente teorema:

\textbf{Teorema}: Sia $G = (V, E)$ un grafo non diretto, connesso e pesato. Sia $A$ un
sottoinsieme di $E$ incluso in una qualche MST di $G$, sia $(S, V \setminus S)$ un
taglio che rispetta $A$, e sia $(u, v)$ un \textit{light edge} per $(S, V \setminus S)$.
Allora $(u, v)$ è \textit{safe} per $A$.

\subsection{Definizione di disuguaglianza triangolare}

La disuguaglianza triangolare è definita come
\begin{equation}
    \forall u, v, w \in V, \textnormal{ vale che } c(u, v) \le c(u, w) + c(w, v)
\end{equation}

dove $c$ è la \textit{funzione di costo}. Questo implica che il cammino con un lato $c(u, v)$ ha un costo
minore o uguale al costo del cammino con due lati $c(u, w, v)$. \\

Questa definizione è necessaria alla definizione delle proprietà degli algoritmi \textit{nearest neighbor} e di 2-approssimazione.