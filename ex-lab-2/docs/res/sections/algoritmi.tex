\section{Descrizione degli algoritmi}

\subsection{Struttura dati per il grafo}

La struttura dati per il grafo è stata implementata nel seguente
modo:

\subsection{Held e Karp}

\subsubsection{Introduzione}


\subsubsection{Implementazione}

\subsubsection{Ottimizzazioni implementate}

\subsection{Nearest Neighbor}

\subsubsection{Introduzione}

\subsubsection{Implementazione}

\subsubsection{Ottimizzazioni implementate}

\subsection{2-approssimato}

\subsubsection{Introduzione}

\textbf{Definizione di MST}

Sia $V$ l'insieme dei nodi che costituiscono il grafo pesato $G$ e sia $E$ la collezione dei lati di tale
grafo. Ai fini delle analisi della complessità degli algoritmi, sia $|V| = n$ e $|E| = m$.

Un \textit{minimum spanning tree} è un
sottoinsieme dei lati $E$ di un grafo $G$ non orientato connesso e pesato sui lati che
collega tutti i vertici insieme, senza alcun ciclo e con il minimo peso totale del
lato possibile. Cioè, è uno spanning tree la cui somma dei pesi dei bordi è la più
piccola possibile.

Un \textit{minimum spanning tree} $T = (V, E')$ è un albero, il cui insieme dei lati $E'$ è un
sottoinsieme dei lati $E$ di un grafo $G = (V, E)$ non orientato, connesso e
pesato, che collega tutti i vertici $V$, la cui somma dei pesi dei lati è la minima.

L'algoritmo generico per determinare un MST è:
\begin{verbatim}
    A = empty_set
    while A doesn't form a spanning tree
        find an edge (u,v) that is safe for A
        A = A U {(u,v)}
    return A
\end{verbatim}

Si forniscono alcune definizioni per gli MST:
\begin{enumerate}
    \item un \textbf{taglio} $(S, V \setminus S)$ di un grafo $G = (V, E)$ è una partizione
    di $V$;
    \item un lato $(u, v) \in E$ \textbf{attraversa il taglio} $(S, V \setminus S)$ se
    $u \in S$ e $v \in V \setminus S$ o viceversa;
    \item un taglio \textbf{rispetta} un insieme $A$ di lati se nessun lato di $A$ attraversa
    il taglio;
    \item dato un taglio, il lato che lo attraversa di peso minimo si chiama \textbf{light edge}.
\end{enumerate}

Per determinare se un lato è \textbf{safe}, si sfrutta il seguente teorema:

\textbf{Teorema}: Sia $G = (V, E)$ un grafo non diretto, connesso e pesato. Sia $A$ un
sottoinsieme di $E$ incluso in una qualche MST di $G$, sia $(S, V \setminus S)$ un
taglio che rispetta $A$, e sia $(u, v)$ un \textit{light edge} per $(S, V \setminus S)$.
Allora $(u, v)$ è \textit{safe} per $A$.

\subsubsection{Implementazione}

\subsubsection{Ottimizzazioni implementate}

