\subsection{Karger e Stein}
\label{karger_stein_section}

\subsubsection{Introduzione all'algoritmo di Karger}
I passi svolti dall'algoritmo di Karger sono:
\begin{enumerate}
    \item Viene scelto a caso un lato;
    \item Si \textit{contraggono} (si veda \ref{definizione_contrazione}) i due vertici 
    relativi, eliminando tutti i lati incidenti su entrambi;
    \item Si ripetono questi due passi finchè restano solo \textit{due} vertici;
    \item Quando sono rimasti soltanto due vertici, si restituiscano i lati che  
    connettono i due vertici.
\end{enumerate}

\subsubsection{Introduzione all'algoritmo di Karger-Stein}
L'algoritmo di Karger-Stein è un algoritmo randomizzato ricorsivo che fa parte della 
categoria di algoritmi Monte Carlo. La versione di questo algoritmo è in grado di 
offrire delle prestazioni migliori rispetto all'algoritmo di Karger e può essere 
applicato per grafi pesati (si veda la descrizione del problema 
\ref{descrizione_min_cut}). Per fare ciò, l'operazione di \verb|full_contraction| 
deve essere ridefinita, in modo tale che:
\begin{enumerate}
    \item Deve essere scelto un lato con probabilità proporzionale al peso del lato 
    \item stesso;
    \item Sia possibile scegliere il lato da contrarre in tempo lineare;
    \item Effettuare la contrazione in tempo lineare.
\end{enumerate}

Riprendendo quanto illustrato precedentemente (si veda 
\ref{struttura_dati_karger_stein}), la struttura dati \verb|KargeGraph| 
include al suo interno altre due strutture dati:
\begin{enumerate}
    \item La matrice di adiacenza \textit{pesata} \verb|W|
    $n \times n$, tale che: 
    \[
        W[u,v] = w(u,v) \textnormal{ se } (u,v) \in \mathcal{E} 
        \textnormal{, altrimenti } 0
    \]
    \item Il vettore dei \textbf{gradi pesati} dei vertici:
    \[
        D[u] = \sum_{v \in \mathcal{V}} W[u,v]
    \]
\end{enumerate}

La procedura di \verb|full_contraction| viene scomposta in due fasi: 
nella prima fase viene determinato il lato e successivamente avviene la 
contrazione.

Questo algoritmo funziona con una probabilità molto bassa. Tuttavia tale probabilità 
può essere significativamente incrementata andando a ripetere questo processo un certo 
numero di volte.

L'algoritmo di Karger-Stein è composto dalle seguenti fasi:
\begin{enumerate}
    \item \textbf{Random select}
    \item \textbf{Edge select}
    \item \textbf{Contract edge}
    \item \textbf{Contract}
    \item \textbf{Recursive contract}
\end{enumerate}

\subsubsection*{Random select}
Si vuole scegliere un lato con una probabilità 
proporzionale al peso del lato stesso. Per fare ciò, si introducono i 
\textbf{pesi cumulativi}. Dati $m$ lati $e_1, e_2, \cdots , e_m$ di peso 
$w_1, w_2, \cdots , w_m$, i pesi cumulativi sono così definiti:
\[
    C[k] = \sum_{i = 1}^{k} w_i
\]
ovvero, alla posizione $k$ del \textit{vettore dei pesi cumulativi} $C$, si avrà 
la somma dei pesi dei primi $k$ lati. Quindi, in $C[m]$ si avrà la somma dei pesi 
di tutti i lati del grafo $\mathcal{G}$. A questo punto:
\begin{enumerate}
    \item Si determina un valore intero $0 \le r \le C[m]$ con 
    \textbf{probabilità uniforme};
    \item \label{scelta_lato} Si utilizza la \textit{ricerca binaria} per 
    determinare il lato $e_i$ tale che $C[i - 1] \le r \le C[i]$:
    \[
        Pr[e_i \textnormal{ viene scelto}] = Pr[C[i - 1] \le r \le C[i]] = \frac{C[i] - C[i - 1]}{C[m]} = \frac{w(e_i)}{\sum_{e \in \mathcal{E}} w(e)}
    \]
    quindi, la probabilità è il peso del lato diviso la somma dei pesi totali del 
    grafo $\mathcal{G}$. Di conseguenza, è proporzionale al peso del lato $w(e_i)$.
\end{enumerate}
L'input di questa procedura può essere un \textit{qualsiasi} vettore di pesi 
cumulativi, non necessariamente di lati.

Dal punto di vista computazionale:
\begin{enumerate}
    \item Costruzione del vettore $C$: $\mathcal{O}(m)$;
    \item Scelta del valore $r$: $\mathcal{O}(1)$;
    \item Determinare il lato $e_i$ con la ricerca binaria in $C$: 
    $\mathcal{O}(log(m))$.
\end{enumerate}
Quindi, la complessità totale di questa procedura è $\mathcal{O}(m)$. Nel caso 
peggiore, ovvero, quando il grafo è \textit{denso}, si ha $m \le \mathcal{O}(n^2)$. 
Di conseguenza la complessità di questa procedura nel caso di un grafo denso è di 
$\mathcal{O}(n^2)$. Questo risulta essere un problema, in quanto andrebbe ad 
incrementare significativamente la complessità totale dell'algoritmo. Per tale 
motivo, al posto di determinare il lato, come definito al punto \ref{scelta_lato}, 
si vuole determinare la posizione del valore $r$ nel vettore $C$.

\subsubsection*{Edge select}
Per i motivi espressi al punto 
precedente, tramite la procedura \verb|random_select| si andrà a determinare il 
vertice $u$ di partenza e poi il vertice $v$ di arrivo, in modo che la probabilità 
di scegliere un lato sia proporzionale al peso del lato stesso. La procedura 
\verb|edge_select| prende in input \verb|W| e il vettore dei gradi pesati dei 
vertici \verb|D| e svolge le seguenti operazioni:
\begin{enumerate}
    \item Si sceglie un vertice $u$ con una probabilità proporzionale a $D[u]$:
    \begin{enumerate}
        \item Si costruisce il vettore dei pesi cumulativi di $D[u]$;
        \item Viene chiamata la procedura \verb|random_select| per scegliere il 
        primo vertice;
    \end{enumerate}
    \item Si sceglie un vertice $v$ con una probabilità proporzionale a $W[u,v]$:
    \begin{enumerate}
        \item Si costruisce il vettore dei pesi cumulativi di $W[u,v]$;
        \item Viene chiamata la procedura \verb|random_select| per scegliere il 
        secondo vertice;
    \end{enumerate}
    \item Si ritorna il lato $(u,v)$.
\end{enumerate}
La complessità di questa procdeura è $\mathcal{O}(n)$ in quanto $D$ e $W[u,v]$ 
hanno una dimensione pari a $\mathcal{O}(n)$ e di conseguenza \verb|random_select| 
opererà in $\mathcal{O}(n)$ in entrambi i casi (si ricordi che \verb|W| è 
$n \times n$). Così facendo, la procedura non ha una dipendenza quadratica legata 
al numero di lati del grafo $\mathcal{G}$, ma ha una dipendenza lineare legata al 
numero di vertici del grafo $\mathcal{G}$.

\[
    Pr[(u,v) \textnormal{ scelto}] = Pr[u \textnormal{ scelto}] \cdot 
    Pr[v \textnormal{ scelto} | u] + Pr[v \textnormal{ scelto}] \cdot 
    Pr[u \textnormal{ scelto} | v]
\]
\[
    = \frac{D[u]}{\sum_v D[v]} \cdot \frac{W[u,v]}{D[u]} + \frac{D[v]}{\sum_v D[v]} 
    \cdot \frac{W[v,u]}{D[v]} =
    \frac{2 \cdot W[u,v]}{\sum_v D[v]} \textnormal{ ed è proporzionale al peso del lato } W[u,v]
\]

\subsubsection*{Contract edge}
Sulla matrice \verb|W| si va 
ad azzerare la riga e la colonna corrispondente al vertice $v$ (che verrà 
contratto). Azzerare $v$ significa \textit{eliminare} il vertice dal grafo. Si va 
ad aggiornare conseguentemente anche la matrice \verb|W| in modo da avere soltanto 
i pesi sulla matrice che corrispondono ai vertici restanti dopo la contrazione. 
Si illustra di seguito lo pseudo-codice:
\begin{verbatim}
    function contract_edge(u, v)
        D[u] = D[u] + D[v] - 2W[u,v]
        D[v] = 0
        W[u,v] = W[v,u] = 0
        for w in V, eccetto u e v do
            W[u,w] = W[u,w] + W[v,w]
            W[w,u] = W[w,u] + W[w,v]
            W[v,w] = W[w,v] = 0
\end{verbatim}
Questa procedura opera in tempo $\mathcal{O}(n)$.

\subsubsection*{Contract}
Si definisce la procedura di contrazione che opera in 
$\mathcal{O}(n^2)$. Questa procedura ritorna una contrazione di $k$ vertici del 
grafo $\mathcal{G}$ rappresentato con la matrice \verb|W| ed il vettore \verb|D|.
Si illustra di seguito lo pseudo-codice:
\begin{verbatim}
    function contract(G = (D,W), k)
        n = numero di vertici in G
        for i = 1 to n - k do
            (u,v) = edge_select(D,W)
            contract_edge(u,v)
        return D,W
\end{verbatim}
Inizialemente, quando il grafo $\mathcal{G}$ ha $n$ vertici, la probabilità di 
sbagliare, ovvero, di contrarre un lato che fa parte del min-cut, è molto bassa. 
Nel momento in cui si arriva ad avere tre veritici nel grafo $\mathcal{G}$, la 
probabilità di scegliere il lato sbagliato è pari a $\frac{2}{3}$.

\subsubsection*{Recursive contract}
Si effettua un'unica fase di contrazione per i primi vertici, in modo tale 
che la probabilità di sbaglaire sia inferiore del 50\%. Per la fase successiva, 
cioè quando si va a contrarre un grafo piccolo, dove la probabilità di errore è 
maggiore del 50\%, anzichè contrarre in maniera randomica, si fanno 
\textit{due tentativi} di contrazione, ottenendo due soluzioni distinte e sceglieno 
quella migliore (ovvero, quella di peso minimo). In questo modo anche se la probabilità 
di errore in questa seconda fase è più alta, poichè si sceglie la migliore soluzione 
tra due scelte randomiche, la probabilità di errore rimane comunque bassa.

Nel caso di \verb|full_contraction| dell'algoritmo di Karger, la probabilità di errore 
è minore di $\frac{2}{n^2}$, mentre nel caso di \verb|contract| la probabilità di 
errore è minore di $(\frac{k}{n})^2$. Quindi, per stabilire il numero di contrazioni 
$k$, si vuole che $(\frac{k}{n})^2 \le \frac{1}{2}$. Risolvendo la disequazione, si 
ottiene che $k = \frac{n}{\sqrt2}$.

Si illustra di seguito lo pseudo-codice:
\begin{verbatim}
    function recursive_contract(G = (D,W))
        n = numero di vertici in G 
        if n <= 6 then
            G' = contract(G,2)
            return del peso dell'unico lato (u,v) in G'
        t = ceil(n/sqrt(2) + 1)
        for i = 1 to 2 do
            G_i = contract(G,t)
            w_i = recursive_contract(G_i)
        return min(w_1,w_2)
\end{verbatim}
La complessità di questa procedura è $\mathcal{O}(n^2 \cdot log(n))$ e trova un 
taglio minimo con probabilità o uguale a $\frac{1}{log(n)}$. Quindi, con $log^2(n)$ 
ripetizioni di \verb|recursive_contract|, la probabilità di errore è minore o uguale 
a $\frac{1}{n}$. In conclusione, la complessità computazionale dell'algoritmo di 
Karger-Stein è $\mathcal{O}(n^2 \cdot log^3(n))$.

\[
    \mathcal{T}(n) = \mathcal{O}(n^2) \textnormal{ se } n \le 6 \textnormal{, altrimenti } 2 \cdot \Bigl(\mathcal{O}(n^2) + \mathcal{T}\Bigl(\frac{n}{\sqrt2} + 1\Bigr)\Bigr)  
\]
ottenendo così
\[
    \mathcal{T}(n) = \mathcal{O}(n^2 \cdot log(n))  
\]

Per quanto riguarda la probabilità di errore:
\[
    \mathcal{P}(n) = 1 - \Bigl(1 - \frac{1}{2} \cdot \mathcal{P}\Bigl(\frac{n}{\sqrt2} + 1\Bigr)\Bigr)^2
\]
per analizzare questa equazione, si passa alla probabilità di aver ottenuto un valore 
corretto alla $k$-esima chiamata ricorsiva:
\[
    \mathcal{P}(k) = 1 - \Bigl(1 - \frac{1}{2} \cdot \mathcal{P}(k - 1)\Bigr)^2 \simeq \frac{1}{k}
\]
Qualora il numero delle chiamate ricorsive è logaritmico rispetto al numero di vertici, 
si ottiene che:
\[
    \mathcal{P}(n) \le c \cdot \frac{1}{log(n)}
\]

\subsubsection{Implementazione}
Per implementare l'algoritmo di Karger e Stein in Python, abbiamo realizzato una classe 
denominata \verb|KargerStein|, che definisce una serie di metodi. I metodi rispecchiano 
le varie fasi descritte nella sezione precedente:
\begin{enumerate}
    \item \verb|random_select|: permette di scegliere randomicamente un certo valore 
    $r$ e restituisce, tramite la ricerca binaria, la posizione di tale valore nel 
    vettore dei pesi cumulativi $C$;
    \item \verb|edge_select|: permette di sceglie un lato del grafo $\mathcal{G}$ in 
    tempo lineare rispetto al numero di vertici del grafo. Per determinare i vertici, 
    questa procedura utilizza \verb|random_select|;
    \item \verb|contract_edge|: dati $u$ e $v$, effettua la contrazione del lato scelto;
    \item \verb|contract|: definisce l'operazione completa di contrazione, aggiornando 
    tutte le strutture dati;
    \item \verb|recursive_contract|: corrisponde al metodo \verb|full_contraction| 
    dell'algoritmo di Karger, però ottenendo un vantaggio dal punto di vista 
    computazionale e la possibilità di essere applicato per grafi pesati;
    \item \verb|algorithm|: questa procedura ricorsiva richiama
    \verb|recursive_contract| un numero sufficiente di volte per poter garantire la 
    correttezza in alta probabilità.
\end{enumerate}

Il metodo \verb|measurements| è stato utilizzato per effettuare le misurazioni per il 
calcolo della complessità computazionale, per determinare i \textit{discovery time} e 
per effettuare altre analisi che verranno illustrate successivamente.

Complessivamente, l'implementazione effettiva dell'algoritmo non si discosta molto 
dallo pseudo-codice presentato a lezione. Differisce soltanto per i dettagli 
implementativi dovuti al linguaggio utilizzato, come la costruzione dei vettori e 
delle strutture dati necessarie.

\subsubsection{Ottimizzazioni implementate}
Per questo algoritmo non sono state attuate delle particolari ottimizzazioni. Tuttavia, 
per poter incrementare significativamente il tempo di esecuzione dell'algoritmo, sono 
state applicate alcune \textit{accortezze}. Proprio per tale motivo, è stata realizzata 
una struttura dati dedicata (\verb|KargerGraph|), in modo da mantenere soltano le 
variabili e le strutture dati di supporto strettamente necessarie. Il motivo per cui il 
vettore dei gradi pesati dei vertici è incluso nella struttura dati \verb|KargerGraph| 
è dovuto per il semplice motivo di mantenere pulito il codice del metodo 
\verb|algorithm|. Per poter implementare quest'ultimo metodo, era necessario che ogni 
volta che venisse effettuata una chiamata al metodo \verb|recursive_contract|, a 
tale metodo fosse passato una \textit{copia} del grafo $\mathcal{G}$ di partenza. 
Per fare ciò, il linguaggio Python mette a disposizione il metodo \verb|deepcopy|, 
un'operazione computazionalmente onerosa. Durante lo sviluppo dell'algoritmo, abbiamo 
effetuato la \verb|deepcopy| della struttura dati \verb|KargerGraph|. Tuttavia, abbiamo 
notato che il tempo di esecuzione dell'algoritmo risultava essere troppo elevato. 
Pertanto, abbiamo deciso di effettura la \verb|deepcopy| delle singole strutture dati 
contenute in \verb|KargerGraph|, ovvero, la matrice di adiacenza pesata \verb|W|, il 
vettore dei gradi pesati dei vertici \verb|D| e il numero dei vertici \verb|n_vertices|
del grafo $\mathcal{G}$. Così facendo, abbiamo notato che l'algoritmo era in grado di 
eseguire le varie istanze in un tempo ragionevole.

L'unica importante variazione che ha l'implementazione rispetto allo pseudo-codice è 
l'introduzione di una soglia (\textit{thershold}) sul tempo di esecuzione dell'algoritmo. 
Come suggerito dalla consegna, l'algoritmo potrebbe impiegare molto tempo per computare 
le istanze più grandi. Per tale motivo, il codice originale è stato modificato in modo 
appropriato per poter implementare l'interruzione dell'esecuzione dell'algoritmo al 
superare di una certa soglia temporale. Se il tempo di esecuzione è maggiore della 
soglia impostata, l'algoritmo ritorna la soluzione migliore che è riuscito a determinare 
fino a quel momento. Quindi, tale risultato \textit{potrebbe} essere una soluzione 
\textit{subottimale} del problema.
