\section{Descrizione degli algoritmi}

\subsection{Struttura dati per il grafo}

La struttura dati per il grafo è stata implementata nel seguente
modo:
\begin{enumerate}
    \item \verb|V| è un insime di nodi;
    \item \verb|E| è una lista di lati;
    \item \verb|graph| rappresenta la \textit{lista di adiacenza}.
    Gli indici per accedere alla mappa sono rappresentati dai vertici.
    Ogni cella della mappa punta ad una lista di coppie di valori
    (il vertice a cui è collegato e il peso del lato che li
    congiunge).
\end{enumerate}

I metodi implementati sono:
\begin{enumerate}
    \item \verb|add_vertex|: aggiunge un vertice al grafo;
    \item \verb|add_edge|: aggiunge un lato al grafo;
    \item \verb|remove_edge|: rimuove un lato dal grafo.
\end{enumerate}
Queste operazioni sono state implementate in modo da avere una
complessità computazionale costante.

\subsection{Kruskal naive}

\subsubsection{Introduzione}

La versione naive dell'algoritmo di Kruskal ha una complessità
computazionale pari a $\mathcal{O}(mn)$. L'idea alla base di questo
algoritmo è quella di ordinare i lati in ordine crescente rispetto al
loro peso $w$. Una volta ordinati, per ogni lato si controlla se
aggiungendolo al grafo temporaneo questo crei un ciclo. Se crea un
ciclo allora tale lato non verrà inserito, altrimenti il lato farà
parte dell'MST $T$.

Ordinando i lati in base al loro peso e controllando che l'inserimento
di un lato non crei un ciclo nel grafo, si otterà un albero la cui
somma dei suoi lati sarà minima.

Si illustri lo pseudocodice:
\begin{verbatim}
    Kruskal-Naive(G)
        A = empty_set
        for each vertex v in G.V
            make-set(v)
        sort the edges of G.E into increasing order by weight w
        for each edge (u, v) in G.E, taken in increasing order by weight
            if find-set(u) != find-set(v)
                A = A U {(u, v)}
                union(u, v)
        return A
\end{verbatim}

\subsubsection{Implementazione}

Nella nostra implementazione, abbiamo utilizzato l'algorimo di
ordinamento \textit{merge sort}, che cha una complessità
computazionale pari a $\Theta(n log(n))$, per ordinare i lati
in ordine crescente rispetto al loro peso.

Il Minimum Spanning Tree viene salvato in una struttura dati
\verb|Graph|. Il MST può essere esaminato andando ad
esplorare la lista di adiacenza \verb|graph| o la lista dei
lati \verb|E|.

Il controllo per la ciclicità viene implementato con una
versione modificata della \textit{Depth First Search}
(\textit{DFS}).

\subsubsection{Ottimizzazioni implementate}

Per ottimizzare l'algoritmo abbiamo apportato delle opportune
modifiche. A partire dal metodo \verb|kruskal_naive|, ad
ogni iterazione del ciclo, si controlla se il numero di lati
del grafo $A$ (quello che conterrà il MST) ha $n - 1$ lati,
dove $n$ è il numero di vertici del grafo $G$, fornito in input.
Quando si raggiunge questa uguaglianza, significa che è stato
costruito il MST $T$, pertanto l'algoritmo può terminare.
Questa prima modifica permette di evitare di fare iterazioni
inutili che non contribuiranno alla costruzione del MST.

La seconda ottimizzazione implementata riguarda il metodo
\verb|_is_acyclic|. Un primo controllo riguarda i vertici
connessi dal lato. Se i due vertici sono uguali significa
che il lato in questione crea un \textit{self-loop}. Con
questo controllo siamo in grado eliminare i self-loop in
tempo costante, senza dover effettuare una chiamata a DFS per verificare
che l'aggiunta di tale lato crei un ciclo nel grafo $A$.
Il secondo controllo che è stato implementato in questo metodo
riguarda i vertici $u$ e $v$ che il lato, fornito in input, connette.
Se almeno uno dei due lati non è presente nel grafo $A$, allora
possiamo concludere che tale lato non introdurrà un ciclo nel grafo,
in quanto il lato in questione permetterà di scoprire almeno un nuovo
vertice del MST finale. Questa accortezza ci permette di risparmiare
di effettuare delle chiamate a DFS soprattuto nella fase di avvio
dell'algoritmo. Se invece, sia $u$ che $v$ sono già presenti
nel grafo $A$, allora è necessario fare una chiamata a DFS per verificare
che quel lato non introduca un ciclo nel grafo.

La terza ottimizzazione riguarda DFS. Il metodo che è stato implementato
non rispecchia la versione originale dell'algoritmo. Lo scopo di DFS
è quello di andare a visitare \textit{tutti} i vertici di un certo
grafo. Tuttavia, questo non rappresenta il nostro scopo. Il nostro
intento è quello di cercare, se esiste, un cammino che congiunge $u$ con
$v$. Se tali vertici si trovano in due componenti connesse distinte
(può capitare nel momento in cui si stanno aggiungendo i lati al
grafo $A$), significa che l'aggiunta del lato $(u, v)$ al grafo $A$ non
introdurrà un ciclo. Avviando DFS dal vertice $u$ sul grafo $A$,
desidero verificare se esiste un cammino che mi porta fino a $v$.
Se entrambi i vertici sono nella stessa componente connessa, allora
tale cammino verrà trovato, altrimenti, tale cammino non verrà
trovato perchè non esiste. Quindi, DFS verrà eseguita soltanto sulla
componente connessa di $u$ e non su tutti i vertici del grafo.
Questa versione permette di risparmiare di visitare tutto il grafo
e la sua complessità computazionale diventa $O(m)$.

Queste tre ottimizzazioni permettono di evitare di fare eccissive
chiamate a DFS e di evitare di andare a visitare l'intero grafo
anche quando un cammino da $u$ a $v$ non esiste. La complessità
computazionale rimane sempre $O(mn)$, però l'esecuzione è sensibilmente
molto più veloce rispetto ad un'implementazione senza queste accortezze.

Il metodo \verb|is_there_a_path| serve soltanto per inizializzare
il vettore di booleani che indica se un vertice è stato visitato o meno.
Tale vettore è necessario a DFS per esplorare il grafo (o la
componente connessa in cui vi è $u$).

\subsection{Kruskal con Union-Find}

\subsubsection{Introduzione}

\subsubsection{Implementazione}

\subsubsection{Ottimizzazioni implementate}

\subsection{Prim}