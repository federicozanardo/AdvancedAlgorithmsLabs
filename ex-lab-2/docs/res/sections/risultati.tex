\section{Risposte alle domande}

\subsection{Domanda 1}

\textit{Eseguite i tre algoritmi che avete implementato (Held-Karp, euristica costruttiva e 2-approssimato) sui 13 grafi
del dataset. Mostrate i risultati che avete ottenuto in una tabella come quella sottostante. Le righe della tabella
corrispondono alle istanze del problema. Le colonne mostrano, per ogni algoritmo, il peso della soluzione trovata, il
tempo di esecuzione e l'errore relativo calcolato come $(SoluzioneTrovata - SoluzioneOttima) / SoluzioneOttima$. Potete
aggiungere altra informazione alla tabella che ritenete interessanti.}

%Abbiamo implementato il codice in Python per l'esecuzione dei tre algoritmi su tutto il dataset fornito. I risultati, che sono stati riportati in modo dettagliato e con i relativi pesi anche nella sezione appendice, sono riportati di seguito.


% ==================================================================================================

\subsection{Domanda 2}
\textit{Commentate i risultati che avete ottenuto: come si comportano gli algoritmi rispetti alle varie istanze?
C'è un algoritmo che riesce sempre a fare meglio degli altri rispetto all'errore di approssimazione? Quale dei tre
algoritmi che avete implementato è più efficiente?}


Analizzando i tre algoritmi abbiamo appurato che ...