\section{Caratteristiche del programma e Misurazioni}

\subsection{Caratteristiche del programma}

\subsubsection{Introduzione}

Il programma è stato realizzato interamente in Python e si struttura in 4 cartelle principali che rappresentano: il modulo degli algoritmi (\textit{algorithms}), il modulo delle strutture dati (\textit{data\_structures}), il modulo delle misurazioni (\textit{measurements}) e il dataset contenente i grafi (\textit{dataset}). 

\subsubsection{Installazione e requisiti}

Prima di eseguire il programma è necessario installare le dipendenze presenti nel file \textit{requirements.txt} eseguendo il seguente comando: \texttt{pip install -r requirements.txt}
\\Per l'esecuzione è richiesto l'uso di un terminale Windows, MacOs, Linux o BSD-like con Python 3.x+ e PIP installato.

\subsubsection{Avvio del programma}

Per l'esecuzione del programma è necessario spostarsi nella cartella di progetto ed eseguire il seguente comando: 
\texttt{python3 main.py [--version] [-h] <type> <dataset>} dove:

\begin{itemize}
    \item \textbf{type:} rappresenta il tipo di algoritmo o di serie di esecuzioni che si vuole avviare, nello specifico uno tra i seguenti: 
    \begin{itemize}
        \item \textit{all} (esegue e salva su file le misurazioni singole);
        \item \textit{hk} (esegue le misurazioni con Held-Karp);
        \item \textit{nn} (esegue le misurazioni con Nearest Neighbor);
        \item \textit{2ap} (esegue le misurazioni con 2-approximation).
    \end{itemize} 
    \item \textbf{dataset:} rappresenta una cartella contenente i file dei dataset formattati come da consegna o singolo file di dataset in input. 
    \item \textbf{--version:} è opzionale e mostra la versione del programma in esecuzione. 
    \item \textbf{-h:} è opzionale e mostra un aiuto per i comandi da utilizzare. 
\end{itemize}

\subsubsection{Caratteristiche tecniche del computer per le misurazioni} 

L'esecuzione del programma e le relative misurazioni sono state effettuate sulla seguente macchina:
\begin{itemize}
    \item \textbf{CPU:} Ryzen 9 3900x (12 core / 24 thread), 4.6ghz
    \item \textbf{RAM:} 32 GB DDR4 
    \item \textbf{SSD:} NVME SSD 512 GB
\end{itemize}

\noindent Tutte le misurazioni sono state eseguite \textbf{sequenzialmente} monitorando l'uso di un core dedicato della CPU al 100\% per tutta la durata dell'esecuzione.

\subsection{Introduzione alle misurazioni} 

Le misurazioni sono state effettuate eseguendo i tre algoritmi per ogni grafo del dataset, misurandone il tempo di esecuzione medio; dalla misura dei tempi è stato ovviamente escluso il tempo di caricamento dei dataset negli oggetti \textsc{tsps}.
Prima di ogni esecuzione dei vari algoritmi è stato inoltre disabilitato temporaneamente il \textit{garbage collector} di Python, così da evitare rallentamenti anche minimi nel corso dell'esecuzione. Infine, ciascuno dei due moduli salva in tre file differenti - uno per ogni algoritmo - i risultati del programma.

\subsection{Misurazione}

\subsubsection{Descrizione} 

La misurazione singola è organizzata sequenzialmente eseguendo in base al tipo di algoritmo l'intero dataset. Il metodo \textit{executeSingleGraphCalculus} applica l'algoritmo richiesto al graph in input e procede nella seguente maniera:

\begin{enumerate}
    \item Esegue una volta l'algoritmo applicato al grafo, disabilitando il \textit{garbage collector} e salvando il tempo di esecuzione.
    \item Verifica se il tempo di esecuzione è maggiore o meno a 1s.
    \begin{itemize}
        \item Se il tempo è minore a 1s, esegue nuovamente \(k\) volte l'algoritmo, disabilitando il \textit{garbage collector} e salvando il tempo di esecuzione medio sulle \(k\) esecuzioni, con \(k\) definito come segue:  \[ k = \left\lfloor\frac{10^9 \textrm{ ns}}{\textrm{tempo medio di esecuzione in ns}}\right\rfloor\]
        ossia il numero di ripetizioni applicate all'algoritmo la cui somma arriva a 1s.
        \item Altrimenti, viene mantenuto il tempo rilevato precedentemente con una singola esecuzione.
    \end{itemize}
    \item Esegue il salvataggio in \textit{append} nel file di output in formato \textsc{csv}.
\end{enumerate}


\subsubsection{File di output}

Il file di output viene salvato direttamente alla fine di ogni esecuzione dell'algoritmo mantenendo la seguente formattazione:
\begin{itemize}
    \item Nome del dataset;
    \item Peso finale calcolato;
    \item Tempo di esecuzione in nanosecondi;
    \item Tempo di esecuzione in secondi;
    \item Esecuzioni totali effettuate dell'algoritmo.
\end{itemize}
