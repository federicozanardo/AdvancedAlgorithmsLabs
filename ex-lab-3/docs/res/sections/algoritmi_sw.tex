\section{Descrizione degli algoritmi}

\subsection{Struttura dati per il grafo}

Per comodità nell'implementazione abbiamo realizzato due diverse strutture dati per rappresentare un multigrafo; queste sono simili tra loro ma presentano alcune differenze.

\subsubsection{Struttura dati per Stoer-Wagner}
Per quanto riguarda la struttura dati per l'esecuzione dell'algoritmo Stoer-Wagner, un oggetto \texttt{Graph} è rappresentato nel seguente modo:

\begin{enumerate}
    \item \verb|V| è un insieme di nodi;
    \item \verb|E| è una lista di lati;
    \item \verb|Graph| rappresenta la \textit{lista di adiacenza}.
    Gli indici per accedere alla mappa sono rappresentati dai vertici.
    Ogni cella della mappa punta ad una lista di coppie di valori
    (il vertice a cui è collegato e il peso del lato che li
    congiunge);
    \item \verb|totalVertex| rappresenta il numero di vertici;
    \item \verb|totalEdges| rappresenta il numero di archi;
    \item \verb|datasetName| rappresenta il nome del grafo.
\end{enumerate}
Essendo l'oggetto in questione un multigrafo, la lista di adiacenza e la lista di lati possono contenere più archi tra due nodi aventi pesi diversi.
\noindent I metodi implementati sono:
\begin{enumerate}
    \item \verb|add_vertex|: aggiunge un vertice al grafo;
    \item \verb|add_edge|: aggiunge un lato al grafo;
    \item \verb|remove_edge|: rimuove un lato dal grafo;
    \item \verb|remove_node|: rimuove completamente un vertice dal grafo, ossia rimuove il vertice dalla relativa lista e tutti gli archi a esso collegati;
    \item \verb|totalWeightCost|: restituisce la somma di tutti i lati tra due vertici.
\end{enumerate}

\subsubsection{Struttura dati per Karger-Stein}
\label{struttura_dati_karger_stein}
La struttura dati per quanto riguarda l'algoritmo di Karger-Stein (\verb|KargerGraph|) contiene i seguenti campi:
\begin{enumerate}
  \item \verb|n_edges|: rappresenta il numero di lati del grafo;
  \item \verb|n_vertices|: rappresenta il numero di vertici del grafo;
  \item \verb|W|: rappresenta la \textit{matrice di adiacenza pesata} del grafo;
  \item \verb|D|: rappresenta il vettore dei \textit{gradi pesati} dei vertici del grafo (successivamente verrà illustrato 
  il motivo dell'utilità di questa struttura).
\end{enumerate}

\noindent I metodi implementati per questa struttura dati sono:
\begin{enumerate}
  \item \verb|add_edge|: aggiunge un lato al grafo;
  \item \verb|remove_edge|: rimuove un lato dal grafo;
  \item \label{calculate_weighted_degrees_vertices} \verb|calculate_weighted_degrees_vertices|: permette di calcolare i gradi pesati dei vertici. Tali risultati vengono 
  memorizzati in \verb|D|.
\end{enumerate}

\subsection{Stoer e Wagner}

\subsubsection{Introduzione}

L'algoritmo di Stoer-Wagner è un algoritmo ricorsivo per la risoluzione del problema del minimum cut di grafi non diretti, pesati e con pesi degli archi non negativi. Questo algoritmo è di tipo deterministico, ossia non vi è alcuna componente di randomicità in esso. \\
L'idea base dell'algoritmo è la seguente: 
\begin{itemize}
  \item A ogni fase, l'algoritmo trova un \textit{s-t minimum cut} tra due vertici;
  \item L'algoritmo contrae quindi il grafo rispetto all'arco $\{s,t\}$ al fine di ricercare un taglio diverso da \textit{s-t};
  \item Il taglio di peso minimo trovato in tutte le iterazioni dell'algoritmo è il risultato dell'algoritmo, e coincide con il \textit{minimum cut} del grafo.
\end{itemize}

\subsubsection{Algoritmo}

L'algoritmo è composto da due funzioni:
\begin{itemize}
  \item \texttt{GlobalMinCut}: funzione ricorsiva che si occupa di chiamare \texttt{stMinCut} e di restituire il \textit{minimum cut} globale di peso minore;
  \item \texttt{stMinCut}: funzione che si occupa di trovare un \textit{s-t minimum cut}.
\end{itemize}

\noindent Viene qui riportato lo pseudocodice per entrambe le funzioni.

\begin{verbatim}
  function GlobalMinCut(G=(V,E,w))
    if V = {a,b} then
      return({a}, {b})
    else
      (C1, s, t) <- stMinCut(G)
      C2 <- GlobalMinCut(G/{s,t})
      if w(C1) <= w(C2) then
        return C1
      else
        return C2
\end{verbatim}

\begin{verbatim}
  function stMinCut(G=(V,E,w))
    A <- {a}
    while A != V do
      trova v in V che massimizza w(A,{v})
      A <- A unito {v}
    siano s e t gli ultimi due vertici aggiunti ad A
    return (V-{t},{t}), s, t
\end{verbatim}

Poiché la ricerca del nodo che massimizza \(w(A,\{v\})\) può essere un'operazione computazionalmente pesante, complice anche la necessità di iterare sull'intera lista di nodi, l'algoritmo può essere facilmente ottimizzato tramite l'utilizzo di una coda di priorità; questa infatti permette di ottenere il nodo di peso massimo in tempo costante (ad eccezione delle operazioni per il mantenimento della proprietà del max-heap, le quali hanno complessità logaritmica). La funzione \texttt{stMinCut} può essere quindi modificata come segue.

\newpage
\begin{verbatim}
  function stMinCut(G=(V,E,w))
    Q <- empty
    for all u in V do
      key[u] <- 0
      Insert(Q, u, key[u])
    s, t <- null
    while Q != empty do
      u <- ExtractMax(Q)
      s <- t; t <- u
      for all (u,v) in E do
        if v in Q then
          key[v] <- key[v] + w(u,v)
          IncreaseKey(Q, v, key[v])
    return (V-{t},{t}), s, t
\end{verbatim}

La complessità dell'algoritmo è fortemente legata alla struttura dati che viene utilizzata. La funzione \texttt{stMinCut}, infatti, ha complessità $O(m\cdot logn)$ se implementata tramite \textit{MaxHeap} e $O(m+n \cdot logn)$ se implementata tramite \textit{Fibonacci Heap}. \\
Di conseguenza la funzione \texttt{GlobalMinCut}, e quindi l'algoritmo, ha complessità $O(m\cdot n\cdot logn)$ se implementato tramite \textit{MaxHeap} e $O(m\cdot n+n^2 \cdot logn)$ se implementato tramite \textit{Fibonacci Heap}. Nel caso specifico della nostra implementazione, avendo usato la struttura dati \textit{MaxHeap}, ci attendiamo che la complessità teorica sia $O(m\cdot n \cdot logn)$.

\subsubsection{Implementazione}

Per implementare l'algoritmo Stoer e Wagner in Python abbiamo realizzato una classe denominata \texttt{StoerWagner} che definisce una serie di metodi per la corretta esecuzione dell'algoritmo.
Nello specifico: 

\begin{itemize}
    \item \texttt{algorithm}, che si occupa di mandare in esecuzione \texttt{globalMinCut} sul grafo di input;
    \item \texttt{stMinCut}, che cerca un min-cut \((s,t)\) sfruttando un \textit{max-heap} come \textit{priority queue} da cui estrarre i nodi con peso dei lati maggiore, individuando così i vertici $s$ e $t$ da ritornare;
    \item \texttt{globalMinCut}, che esegue la ricerca ricorsivamente di un min-cut globale a partire dal grafo di input e, nel caso in cui il grafo contenga solo due nodi, restituisca come caso base i nodi presenti come tupla.
    \item \texttt{weightMinCut}, usato per ritornare il peso del vertice in input a partire dai pesi dei lati;
    \item \texttt{contractGraph}, usato contrarre il grafo in modo iterativo rispetto ai vertici \textit{s} e \textit{t} in input. 
\end{itemize}

Le implementazioni fanno uso della struttura dati \texttt{graph} precedentemente menzionata, attraverso cui viene costruita la lista di adiacenza, nonché il vettore di vertici del grafo in input. Rispetto a quanto riportato nello pseudocodice, inoltre, abbiamo fatto uso delle tuple per conservare i pesi calcolati con \texttt{key[v]} per i singoli vertici, che poi vengono recuperati da \texttt{weightMinCut} per ritornare il peso del min-cut, senza necessità di doverlo ricalcolare.

\subsubsection{Ottimizzazioni implementate}

Per quanto concerne le ottimizzazioni, abbiamo evitato - a meno della prima e unica invocazione di \texttt{algorithm} - la copia profonda di tutto \texttt{Graph}, preferendo la sola copia della lista di vertici, dal momento che questa era quella maggiormente soggetta ad alterazioni. La lista di vertici è stata poi costantemente passata tra i vari metodi di classe e questo ha velocizzato in buona parte il passaggio di variabili, nonché ridotto il tempo totale di esecuzione dell'algoritmo stesso. Inoltre, abbiamo preferito l'uso della copia profonda manuale rispetto a quella nativa del linguaggio, inserendo gli \(n-1\) vertici in una nuova lista, poi passata come risultato dei metodi. 
Al fine di velocizzare la parte di aggiornamento dei pesi dei nodi nel \textit{max-heap}, abbiamo riscritto la funzione \texttt{searchAndUpdateWeight} in \texttt{increaseKey} come metodo a sé stante, garantendo così la riduzione di una buona percentuale di tempo impiegato per l'aggiornamento del peso dei nodi, svolgendo così solo l'operazione di \texttt{heapifyUp}. Inoltre, sempre nel \textit{max-heap}, abbiamo recuperato una vecchia ottimizzazione consistente nell'utilizzo di una mappa, concretizzata in una variabile di tipo \texttt{defaultdict} in Python, che associa a ogni nodo la sua posizione in \textit{list}, ossia la lista dei nodi. Questa implementazione permette di ricercare la presenza di un vertice nello Heap in tempo costante, e ciò è particolarmente utile perché la presenza del vertice deve essere verificata due volte ad ogni iterazione del ciclo for dell'algoritmo (nella condizione if e durante l'aggiornamento dello heap). Se non avessimo adottato tale ottimizzazione, queste operazioni avrebbero avuto una complessità lineare, andando così ad aumentare polinomialmente la complessità computazionale dell'algoritmo.
L'insieme di tutte queste piccole ottimizzazioni ci ha permesso di non dover inserire una soglia di tempo massima entro cui l'algoritmo doveva eseguire.