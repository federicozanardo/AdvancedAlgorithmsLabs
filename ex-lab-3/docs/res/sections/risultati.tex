\section{Risposte alle domande}

\subsection{Domanda 1}
\textit{Misurate i tempi di calcolo dell'algoritmo deterministico di Stoer e Wagner sui grafi del dataset. Mostrate i risultati con un grafico che mostri la variazione dei tempi di calcolo al variare del numero di vertici nel grafo. Confrontate i tempi misurati con la complessità asintotica dell'algoritmo. \\
Nelle istanze più grandi, il tempo di calcolo necessario per completare l'esecuzione potrebbe risultare eccessivo. In questo caso utilizzate un timeout, riportando nei risultati che l'algoritmo non riesce ad ottenere un risultato in tempo utile.}

Abbiamo implementato il codice in Python per l'esecuzione dell'algoritmo di Stoer-Wagner su tutto il dataset fornito. I risultati, che sono stati riportati in modo dettagliato con i relativi pesi anche nella sezione appendice, sono riportati di seguito.

\begin{figure}[H]
	\centering
	\includegraphics[width=0.85\textwidth]{res/images/single/stoerwagner}
	\caption{Complessità di Stoer-Wagner con \textit{k} esecuzioni ripetute per ogni quartetto di grafi con uguale numero di nodi.}
	\label{fig:stoerwagner}
\end{figure}

Nel grafico appena illustrato (fig. \ref{fig:stoerwagner}) è riportata la complessità computazionale attesa (in giallo) ed effettiva (in blu) per l'algoritmo di Stoer e Wagner con più esecuzioni dell'algoritmo. \\
Come si può evincere dall'immagine, la complessità dell'algoritmo da noi implementato ricalca quasi alla perfezione quella teorica. Questo è probabilmente dovuto alle ottimizzazioni che sono state implementate, prime tra tutte l'ottimizzazione della copia profonda e l'utilizzo della mappa degli indici in \texttt{MaxHeap}. Per questo motivo, non è stato necessario impostare un timeout per bloccare l'esecuzione dell'algoritmo.

\subsection{Domanda 2}
\textit{Misurate i tempi di calcolo dell'algoritmo di Karger e Stein sui grafi del dataset, usando un numero di ripetizioni che garantisca una probabilità minore o uguale a $1/n$ di sbagliare. Mostrate i risultati con un grafico che mostri la variazione dei tempi di calcolo al variare del numero di vertici nel grafo. Confrontate i tempi misurati con la complessità asintotica dell'algoritmo. \\
Nelle istanze più grandi, il tempo di calcolo necessario a completare tutte le iterazioni potrebbe risultare eccessivo. In questo caso utilizzate un timeout oppure abbassate il numero di ripetizioni per ottenere tempi di esecuzione ragionevoli.}

\subsection{Domanda 3}
\textit{Misurate il discovery time dell'algoritmo di Karger e Stein sui grafi del dataset. Il discovery time è il momento (in secondi) in cui l'algoritmo trova per la prima volta il taglio di costo minimo.  Confrontate il discovery time con il tempo di esecuzione complessivo per ognuno dei grafi nel dataset.}

\begin{figure}[H]
	\begin{subfigure}{.5\textwidth}
		\centering
		\includegraphics[width=1\textwidth]{res/images/single/karger-stein/discovery-time/karger_stein_confronto_discovery_time_total_time.png}
		\caption{Confronto tra il tempo totale di esecuzione dell'algoritmo e il discovery time per ogni esecuzione.}
		\label{fig:karger_stein_confronto_discovery_time_total_time}
	\end{subfigure}
	\begin{subfigure}{.5\textwidth}
		\centering
		\includegraphics[width=1\textwidth]{res/images/single/karger-stein/discovery-time/karger_stein_rapporto_discovery_time_total_time.png}
		\caption{Rapporto tra il tempo totale di esecuzione dell'algoritmo 
		e il discovery time per ogni esecuzione.}
		\label{fig:karger_stein_rapporto_discovery_time_total_time}
	\end{subfigure}
	\caption{Esecuzione dell'algoritmo di Karger-Stein senza una threshold sul tempo.}
	\label{fig:karger_stein_discovery_time}
\end{figure}

I valori del grafico \ref{fig:karger_stein_rapporto_discovery_time_total_time} 
esprimono la percentuale di tempo impiegato per trovare per la prima 
volta il min-cut, rispetto al tempo totale di esecuzione dell'algoritmo.
Dai grafici è possibile notare che per le istanze più piccole, il rapporto 
tra il discovery time e il tempo totale di esecuzione è maggiore rispetto 
allo stesso rapporto per le istanze più grandi.
L'esecuzione dell'algoritmo con la threshold a 10 secondi (grafici 
\ref{fig:karger_stein_discovery_time_threshold_10s}) impiega, in media, 
impiega un tempo maggiore per determinare il min-cut. È possibile notare che 
l'esecuzione dell'algoritmo senza threshold impiega un tempo maggiore rispetto 
alle esecuzioni con la threshold. Questo perché le istanze più grandi richiedono 
un tempo maggiore per poter determinare il min-cut. Tuttavia, l'esecuzione senza 
threshold non ha prodotto errori rispetto al min-cut ottimo. Il tempo medio del 
discovery time è:
\begin{enumerate}
	\item \textit{Senza threshold}: 4.5 secondi;
	\item \textit{Threshold di 6 secondi}: 2.7 secondi;
	\item \textit{Threshold di 10 secondi}: 3.1 secondi.
\end{enumerate}

Si illustri il discovery time in rapporto al tempo totale di esecuzione dell'algoritmo, in percentuale:
\begin{enumerate}
	\item \textit{Senza threshold}: 7.56\%;
	\item \textit{Threshold di 6 secondi}: 38.42\%;
	\item \textit{Threshold di 10 secondi}: 29.87\%.
\end{enumerate}

\begin{figure}[H]
	\begin{subfigure}{.5\textwidth}
	  \centering
	  \includegraphics[width=1\textwidth]{res/images/single/karger-stein/discovery-time/threshold6/karger_stein_confronto_discovery_time_total_time_threshold_6s.png}
	  \caption{Confronto tra il tempo totale di esecuzione dell'algoritmo e il discovery time per ogni esecuzione.}
	  \label{fig:karger_stein_confronto_discovery_time_total_time_threshold_6s}
	\end{subfigure}
	\begin{subfigure}{.5\textwidth}
	  \centering
	  \includegraphics[width=1\textwidth]{res/images/single/karger-stein/discovery-time/threshold6/karger_stein_rapporto_discovery_time_total_time_threshold_6s.png}
	  \caption{Rapporto tra il tempo totale di esecuzione dell'algoritmo 
	  e il discovery time per ogni esecuzione.}
	  \label{fig:karger_stein_rapporto_discovery_time_total_time_threshold_6s}
	\end{subfigure}
	\caption{Esecuzione dell'algoritmo di Karger-Stein con una threshold sul tempo di 
	6 secondi.}
	\label{fig:karger_stein_discovery_time_threshold_6s}
\end{figure}

\begin{figure}[H]
	\begin{subfigure}{.5\textwidth}
	  \centering
	  \includegraphics[width=1\textwidth]{res/images/single/karger-stein/discovery-time/threshold10/karger_stein_confronto_discovery_time_total_time_threshold_10s.png}
	  \caption{Confronto tra il tempo totale di esecuzione dell'algoritmo e il discovery time per ogni esecuzione.}
	  \label{fig:karger_stein_confronto_discovery_time_total_time_threshold_10s}
	\end{subfigure}
	\begin{subfigure}{.5\textwidth}
	  \centering
	  \includegraphics[width=1\textwidth]{res/images/single/karger-stein/discovery-time/threshold10/karger_stein_rapporto_discovery_time_total_time_threshold_10s.png}
	  \caption{Rapporto tra il tempo totale di esecuzione dell'algoritmo e il 
	  discovery time per ogni esecuzione.}
	  \label{fig:karger_stein_rapporto_discovery_time_total_time_threshold_10s}
	\end{subfigure}
	\caption{Esecuzione dell'algoritmo di Karger-Stein con una threshold sul tempo di 
	10 secondi.}
	\label{fig:karger_stein_discovery_time_threshold_10s}
\end{figure}

\begin{figure}[H]
	\begin{subfigure}{.5\textwidth}
	  \centering
	  \includegraphics[width=1\textwidth]{res/images/single/karger-stein/discovery-time/confronto/karger_stein_confronto_discovery_time_total_time_with_thresholds.png}
	  \caption{Confronto tra il tempo totale di esecuzione dell'algoritmo e il discovery time per ogni esecuzione.}
	  \label{fig:confronto}
	\end{subfigure}
	\begin{subfigure}{.5\textwidth}
	  \centering
	  \includegraphics[width=1\textwidth]{res/images/single/karger-stein/discovery-time/confronto/karger_stein_rapporto_discovery_time_total_time_with_thresholds.png}
	  \caption{Rapporto tra il tempo totale di esecuzione dell'algoritmo e il 
	  discovery time per ogni esecuzione.}
	  \label{fig:rapporto}
	\end{subfigure}
	\caption{Confronto tra le esecuzioni senza threshold, con una threshold sul tempo 
	di 6 secondi e di 10 secondi.}
	\label{fig:karger_stein_discovery_time_thresholds}
\end{figure}

Dai grafici \ref{fig:karger_stein_discovery_time_thresholds} è possibile notare che 
l'esecuzione dell'algoritmo senza una threshold sul tempo ha un discovery 
time maggiore rispetto alle altre due esecuzioni. Questo è dovuto al fatto che per 
istanze più grandi dei grafi, è necessario un tempo maggiore per poter determinare 
il min-cut di tale istanza. Tuttavia, l'esecuzione senza threshold permette di avere 
delle soluzioni più precise (in particolare, l'algoritmo ritorna per ogni istanza la 
soluzione ottima). Osservando il grafo dei rapporti tra discovery time e tempo totale 
di esecuzione dell'algoritmo, l'esecuzione senza la threshold ha dei rapporti che sono 
molto inferiori rispetto alle altre due esecuzioni.

\subsection{Domanda 4}
\textit{Per ognuno dei grafi del dataset, riportate il costo del taglio minimo trovato dai due algoritmi. Per l'algoritmo di Karger e Stein, riportate l'errore relativo calcolato come $SoluzioneTrovata-SoluzioneOttima/SoluzioneOttima$, dove $SoluzioneOttima$ è la soluzione trovata dall'algoritmo deterministico, se esiste.}

L'esecuzione dell'algoritmo di Karger-Stein senza la threshold sul 
tempo di esecuzione ha sempre restiuito la soluzione ottima. Invece, 
imponendo una threshold sul tempo esecuzione è possibile notare dal 
grafico \ref{fig:karger_stein_tassi_di_errore} che l'algoritmo 
comincia a restituire delle soluzioni con una maggiore imprecisione 
a mano a mano che la dimensione dell'istanza aumenta.

Nel grafico è possibile notare che aumentando la threshold 
(da 6 secondi a 10 secondi), l'algoritmo produce delle soluzioni più 
precise, ovvero, il tasso di errore diminuisce. In particolare, 
l'algoritmo diventa più preciso a mano a mano che la dimensione delle 
istanze aumenta.

\begin{figure}[H]
	\centering
	\includegraphics[width=1\textwidth]{res/images/single/karger-stein/tasso-di-errore/karger_stein_tassi_di_errore.png}
	\caption{Variazione del tasso di errore al variare della threshold
	(blu = threshold di 6 secondi, arancione = threshold di 10 secondi).}
	\label{fig:karger_stein_tassi_di_errore}
\end{figure}

% Please add the following required packages to your document preamble:
% \usepackage{longtable}
% Note: It may be necessary to compile the document several times to get a multi-page table to line up properly
\begin{longtable}{llllllll}
	\textbf{Dataset} & \textbf{\begin{tabular}[c]{@{}l@{}}Correct\\ result\end{tabular}} & \textbf{\begin{tabular}[c]{@{}l@{}}Result w/o\\ threshold\end{tabular}} & \textbf{\begin{tabular}[c]{@{}l@{}}Error w/o\\ threshold\end{tabular}} & \textbf{\begin{tabular}[c]{@{}l@{}}Result\\ w/ 6s\end{tabular}} & \textbf{\begin{tabular}[c]{@{}l@{}}Error\\ w/ 6s\end{tabular}} & \textbf{\begin{tabular}[c]{@{}l@{}}Result\\ w/ 10s\end{tabular}} & \textbf{\begin{tabular}[c]{@{}l@{}}Error\\ w/ 10s\end{tabular}} \\
	\endhead
	%
	1 & 3056 & 3056 & 0.0 & 3056 & 0.0 & 3056 & 0.0 \\
	2 & 223 & 223 & 0.0 & 223 & 0.0 & 223 & 0.0 \\
	3 & 2302 & 2302 & 0.0 & 2302 & 0.0 & 2302 & 0.0 \\
	4 & 5152 & 5152 & 0.0 & 4974 & 3.45 & 4974 & 3.45 \\
	5 & 1526 & 1526 & 0.0 & 1526 & 0.0 & 1526 & 0.0 \\
	6 & 1684 & 1684 & 0.0 & 1684 & 0.0 & 1684 & 0.0 \\
	7 & 522 & 522 & 0.0 & 522 & 0.0 & 522 & 0.0 \\
	8 & 2866 & 2866 & 0.0 & 2866 & 0.0 & 2866 & 0.0 \\
	9 & 2137 & 2137 & 0.0 & 2137 & 0.0 & 2137 & 0.0 \\
	10 & 1446 & 1446 & 0.0 & 1446 & 0.0 & 1446 & 0.0 \\
	11 & 648 & 648 & 0.0 & 648 & 0.0 & 648 & 0.0 \\
	12 & 2486 & 2486 & 0.0 & 2486 & 0.0 & 2486 & 0.0 \\
	13 & 1282 & 1282 & 0.0 & 1282 & 0.0 & 1282 & 0.0 \\
	14 & 299 & 299 & 0.0 & 299 & 0.0 & 299 & 0.0 \\
	15 & 2113 & 2113 & 0.0 & 2113 & 0.0 & 2113 & 0.0 \\
	16 & 159 & 159 & 0.0 & 159 & 0.0 & 159 & 0.0 \\
	17 & 969 & 969 & 0.0 & 969 & 0.0 & 969 & 0.0 \\
	18 & 1756 & 1756 & 0.0 & 1756 & 0.0 & 1756 & 0.0 \\
	19 & 714 & 714 & 0.0 & 714 & 0.0 & 714 & 0.0 \\
	20 & 2610 & 2610 & 0.0 & 2610 & 0.0 & 2610 & 0.0 \\
	21 & 341 & 341 & 0.0 & 341 & 0.0 & 341 & 0.0 \\
	22 & 890 & 890 & 0.0 & 890 & 0.0 & 890 & 0.0 \\
	23 & 772 & 772 & 0.0 & 772 & 0.0 & 772 & 0.0 \\
	24 & 1561 & 1561 & 0.0 & 1561 & 0.0 & 1561 & 0.0 \\
	25 & 951 & 951 & 0.0 & 951 & 0.0 & 951 & 0.0 \\
	26 & 424 & 424 & 0.0 & 424 & 0.0 & 424 & 0.0 \\
	27 & 1153 & 1153 & 0.0 & 1153 & 0.0 & 1153 & 0.0 \\
	28 & 707 & 707 & 0.0 & 707 & 0.0 & 707 & 0.0 \\
	29 & 484 & 484 & 0.0 & 484 & 0.0 & 484 & 0.0 \\
	30 & 850 & 850 & 0.0 & 850 & 0.0 & 850 & 0.0 \\
	31 & 1382 & 1382 & 0.0 & 1382 & 0.0 & 1382 & 0.0 \\
	32 & 1102 & 1102 & 0.0 & 1102 & 0.0 & 1102 & 0.0 \\
	33 & 346 & 346 & 0.0 & 346 & 0.0 & 346 & 0.0 \\
	34 & 381 & 381 & 0.0 & 1028 & 169.82 & 381 & 0.0 \\
	35 & 129 & 129 & 0.0 & 129 & 0.0 & 129 & 0.0 \\
	36 & 670 & 670 & 0.0 & 670 & 0.0 & 209 & 68.81 \\
	37 & 1137 & 1137 & 0.0 & 1137 & 0.0 & 1137 & 0.0 \\
	38 & 869 & 869 & 0.0 & 869 & 0.0 & 869 & 0.0 \\
	39 & 868 & 868 & 0.0 & 926 & 6.68 & 868 & 0.0 \\
	40 & 1148 & 1148 & 0.0 & 1869 & 62.8 & 1869 & 62.8 \\
	41 & 676 & 676 & 0.0 & 676 & 0.0 & 676 & 0.0 \\
	42 & 290 & 290 & 0.0 & 290 & 0.0 & 290 & 0.0 \\
	43 & 818 & 818 & 0.0 & 1412 & 72.62 & 818 & 0.0 \\
	44 & 175 & 175 & 0.0 & 175 & 0.0 & 175 & 0.0 \\
	45 & 508 & 508 & 0.0 & 771 & 51.77 & 508 & 0.0 \\
	46 & 904 & 904 & 0.0 & 2320 & 156.64 & 904 & 0.0 \\
	47 & 362 & 362 & 0.0 & 965 & 166.57 & 362 & 0.0 \\
	48 & 509 & 509 & 0.0 & 593 & 16.5 & 509 & 0.0 \\
	49 & 400 & 400 & 0.0 & 400 & 0.0 & 968 & 142.0 \\
	50 & 364 & 364 & 0.0 & 364 & 0.0 & 364 & 0.0 \\
	51 & 336 & 336 & 0.0 & 336 & 0.0 & 336 & 0.0 \\
	52 & 639 & 639 & 0.0 & 639 & 0.0 & 2121 & 231.92 \\
	53 & 43 & 43 & 0.0 & 43 & 0.0 & 43 & 0.0 \\
	54 & 805 & 805 & 0.0 & 1352 & 67.95 & 954 & 18.51 \\
	55 & 363 & 363 & 0.0 & 363 & 0.0 & 363 & 0.0 \\
	56 & 584 & 584 & 0.0 & 761 & 30.31 & 584 & 0.0
	\end{longtable}

\subsection{Domanda 5}
\textit{Commentate i risultati che avete ottenuto: come si comportano gli algoritmi rispetto alle varie istanze? C'è un algoritmo che riesce sempre a fare meglio degli altri? Quale dei due algoritmi che avete implementato è più efficiente? Quale il più preciso nei risultati?}

I risultati ottenuti dai due algoritmi sono perfettamente in linea con quanto ci aspettavamo dal punto di vista della complessità teorica di ognuno. L'andamento delle varie istanze è in linea con la complessità teorica, visto che all'aumentare del numero di nodi aumenta in modo corrispondente il tempo di esecuzione dell'algoritmo da noi implementato. 
Analizzando i tempi di esecuzione dei due algoritmi possiamo affermare che l'algoritmo di Stoer e Wagner risulta essere nettamente più veloce rispetto a quello di Karger e Stein, con un tempo massimo nell'istanza più grande pari a poco più di un secondo (1.12 s per la precisione), mentre il secondo impiega più di 10 minuti (722.44 s) con le ripetizioni w.h.p. Quello di Stoer e Wagner è dunque l'algoritmo migliore su tutti i fronti (nello specifico, tempo computazionale e risultati ottenuti). Confrontandolo con Karger e Stein a livello di efficienza e precisione possiamo vedere che risultati riportati sono corretti senza soglie di tempo impostate, mentre impostando delle soglie di 6 secondi e 10 secondi l'algoritmo ha un tasso di errore minore con soglie maggiori. Questo è triviale, essendo un algoritmo randomico w.h.p., e come ulteriore analisi possiamo vedere anche che quando si arriva a una grandezza pari a 250 nodi gli errori cominciano a manifestarsi, sebbene il tempo di esecuzioni sia comunque superiore (anche in questo caso) rispetto a Stoer e Wagner. 
