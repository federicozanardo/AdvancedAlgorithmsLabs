\section{Introduzione}

\subsection{Descrizione del problema}

In questa relazione illustreremo dei confronti tra tre algoritmi per risolvere un problema
intrattabile, confrontando i tempi di calcolo e la \textit{qualità} delle soluzioni
che si possono ottenere con \textbf{algoritmi esatti} e con \textbf{algoritmi di approssimazione}.
Il problema in questione è il \textbf{Travelling Salesman Problem} (\textbf{TSP} o
\textit{Problema del Commesso Viaggiatore}). Il nome di questo problema deriva dalla sua
rappresentazione: data una rete di città, connesse tra loro tramite strade, si determini il
percorso di minore distanza che un commesso viaggiatore deve fare per visitare tutte le città
\textit{una ed una sola volta}.

Il TSP si può rappresentare con un grafo non orientato, pesato e
\textit{completo} $G = (V,E)$, dove i vertici sono le città ed il peso del lato $(u,v)$
è uguale alla distanza da $u$ a $v$. Risolvere il TSP significa trovare un
\textbf{circuito Hamiltoniano}, ovvero, un ciclo di costo minimo che visita tutti
i vertici \textit{esattamente una volta}.

\subsection{NP-completezza del problema}

\subsubsection{Dimostrazione di NP-completezza}

Prima di tutto dimostriamo che TSP $\in \mathcal{NP}$. Data un’istanza del problema, usiamo
come \textit{certificato} la sequenza degli $n$ vertici del ciclo di peso minimo.
L’algoritmo di verifica del certificato deve controllare che la sequenza contenga ogni
vertice di $G$ esattamente una volta, sommare il peso dei lati e controllare che il peso
totale del ciclo sia inferiore a $k$. Questo controllo si può svolgere in tempo polinomiale
($O(n)$).

Per poter stabilire con precisione la complessità del problema, è necessario prima
“trasformare” tale problema in un \textit{problema di decisione}, aggiungendo un limite $k$
per il peso del ciclo all’input del problema.

\textbf{Definizione di TSP decisionale}: Dato un grafo non orientato, completo e pesato
$G = (V,E)$ e un valore $k > 0$, esiste un ciclo in $G$ che attraversa tutti i vertici una
sola volta di peso inferiore a $k$?

\textbf{Teorema}: TSP è un problema $\mathcal{NP}$-completo.

\textit{Dimostrazione}: è già stato dimostrato che TSP $\in \mathcal{NP}$, quindi si procede nel dimostrare
che TSP è $\mathcal{NP}$-hard. Si mostri una riduzione del problema del circuito Hamiltoniano a TSP.
Prendiamo un grafo non orientato $G = (V,E)$ e costruiamo un'istanza di TSP che ci permetta di risolvere
il problema del circuito Hamiltoniano su $G$. Costruiamo un grafo non orientato e completo $G' = (V, E')$
con gli stessi vertici di $G$ e $E' = \{(u, v) | u, v \in V\}$. Il peso degli archi di $G'$ viene assegnato
come segue

\[
    w(u, v) = 0, \textnormal{ se } \{u, v\} \in E
\]
\[
    w(u, v) = 1, \textnormal{ se } \{u, v\} \notin E
\]

È possibile notare che si può costruire il grafo $G'$ in tempo polinomiale rispetto al numero
di vertici $|V|$ del grafo di partenza $G$. Il grafo $G$ contiene un circuito Hamiltoniano se e
solo se $G'$ ha un ciclo di peso minore o uguale a 0. Supponiamo che $G$ contenga un circuito
Hamiltoniano $h = v_1, \dots, v_n$. Ogni lato che compone $h$ è presente in $E$ e quindi ha peso 0 in $G'$.
Di conseguenza, $h$ è un ciclo di $G'$ di peso uguale a 0. Viceversa, supponiamo che $G'$ contenga un ciclo
semplice $t$ che attraversa tutti i vertici e di peso minore o uguale a 0. Poiché i pesi dei lati
in $G'$ sono solo 0 oppure 1, tutti i lati che compongono $t$ devono avere costo 0. Quindi tutti i
lati del ciclo sono presenti anche in $E$ e $t$ è un circuito Hamiltoniano per $G$.

\subsubsection{Inapprossimabilità per fattori $\rho$ costanti di TSP}

Il TSP è un problema simile al problema del MST: il MST è un cammino di peso minimo
che collega tutti i vertici del grafo $G$, mentre il TSP è un ciclo di peso minimo
che collega tutti i vertici del grafo $G$. Nonostante questa somiglianza tra questi
due problemi, il TSP è un problema molto difficile anche solo da approssimare. Se
esistesse un algoritmo di approssimazione con un $\rho$ costante, allora saprei
risolvere in tempo polinomiale un problema \textit{$\mathcal{NP}$-hard}.

\textbf{Teorema}: se $\mathcal{P} \ne \mathcal{NP}$, non può esistere alcun algoritmo
(polinomiale) di $\rho$-approssimazione per TSP con $\rho = \mathcal{O}(1)$.

\textit{Dimostrazione}: per assurdo, si supponga che esista un algoritmo $A_\rho$
polinomiale di $\rho$-approssimazione per TSP. Si dimostri come costruire $A_{Hamilton}$
che decide il problema del ciclo Hamiltoniano in tempo polinomiale. Sia $I = G =(V,E)$
e $O = $ "$G$ contiene un ciclo Hamiltoniano?". Si effettui una \textit{riduzione}:
\[
    G \rightarrow G' = (V, E') \textnormal{ è completo, dove }
\]
\[
    c(e \in E') = 1 \textnormal{ se } e \in E \textnormal{, altrimenti } c(e \in E') = \rho|V| + 1
\]

Possiamo quindi creare delle rappresentazioni di $G'$ e di $c$ a partire dalle rappresentazioni
di $G$ in tempo polinomiale in $|V|$ ed $|E|$.

Eseguo $A_\rho(G') \rightarrow C \textnormal{ (ciclo), } c(C) \textnormal{(funzione di costo di $C$, $c: V \times V \rightarrow \mathcal{N}$)}$ e si
determina:
\begin{enumerate}
\item $G \in HAMILTON \Rightarrow c(C^*) = |V| \Rightarrow A_\rho$, ritorna un ciclo $C$
con $c(C) \le \rho|V|$;
\item $G \not\in HAMILTON \Rightarrow C$ contiene almeno un lato non in
$G$ (più precisamente in $E$) $\Rightarrow c(C) \ge \rho|V| + 1$. Poichè i lati che non
appartengono a $G$ sono costosi, c'è un \textit{gap} di almeno $\rho |V|$ tra il costo di un cammino
che è Hamiltoniano in $G$ (di costo $|V|$) e il costo di qualsiasi altro cammino (di costo
almeno $\rho |V| + |V|$). Pertanto, il costo di un cammino che non è un ciclo Hamiltoniano
in $G$ è almeno un fattore $\rho + 1$ maggiore del costo di un cammino che è un ciclo
Hamiltoniano in $G$.
\end{enumerate}

È possibile notare che si può costruire il grafo $G'$ in tempo polinomiale rispetto al numero
di vertici $|V|$ del grafo di partenza $G$. Il grafo $G$ contiene un circuito Hamiltoniano se e
solo se $G'$ ha un ciclo di peso minore o uguale a 0. Supponiamo che $G$ contenga un circuito
Hamiltoniano $h = v_1, ..., v_n$. Ogni lato che compone $h$ è presente in $E$ e quindi ha peso 0 in $G'$.
Di conseguenza, $h$ è un ciclo di $G'$ di peso uguale a 0. Viceversa, supponiamo che $G'$ contenga un ciclo
semplice $t$ che attraversa tutti i vertici e di peso minore o uguale a 0. Poiché i pesi dei lati
in $G'$ sono solo 0 oppure 1, tutti i lati che compongono $t$ devono avere costo 0. Quindi tutti i
lati del ciclo sono presenti anche in $E$ e $t$ è un circuito Hamiltoniano per $G$.

\label{tsp_metrico}
\subsection{TSP metrico}

Istanza particolare TSP in cui l’input, in particolare la funzione di costo c, soddisfa la
\textbf{disuguaglianza triangolare}:
\[
\forall u, v, w \in V, \textnormal{ vale che } c(u, v) \le c(u, w) + c(w, v) \Rightarrow
c(<u, v>) \le c(<u, w, v>)
\]

dove $c$ è la \textit{funzione di costo}. Secondo questa definizione, per ogni insieme di tre nodi
possibili del grafo $G$, vale che il costo di un lato $(u, v)$ è al più il costo di un lato $(u, w)$
più il costo di un lato $(w, v)$. Questo significa che il costo del cammino per andare da $u$ a $v$
attraverso l'unico lato che li collega è più conveniente del cammino per andare da $u$ a $w$ e da
$w$ a $v$. Chiamiamo questo problema \textbf{TRIANGLE\_TSP}. Questa restrizione del problema
$\mathcal{NP}$-completo si trova in $\mathcal{P}$.

\textbf{Teorema}: 