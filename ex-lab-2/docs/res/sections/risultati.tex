\section{Risposte alle domande}

\subsection{Domanda 1}

\textit{Eseguite i tre algoritmi che avete implementato (Held-Karp, euristica costruttiva e 2-approssimato) sui 13 grafi
del dataset. Mostrate i risultati che avete ottenuto in una tabella come quella sottostante. Le righe della tabella
corrispondono alle istanze del problema. Le colonne mostrano, per ogni algoritmo, il peso della soluzione trovata, il
tempo di esecuzione e l'errore relativo calcolato come $(SoluzioneTrovata - SoluzioneOttima) / SoluzioneOttima$. Potete
aggiungere altra informazione alla tabella che ritenete interessanti.}

Abbiamo implementato il codice in Python per l'esecuzione dei tre algoritmi su tutto il dataset fornito. Nella tabella che segue sono riportate le soluzioni ottime dei diversi dataset come riferimento per i risultati che abbiamo ottenuto.

\begin{table}[H]
  \centering
  \begin{tabular}{|l|c|c|c|}
  \hline
  \multicolumn{1}{|c|}{\textbf{File}} & \textbf{Descrizione} & \textbf{N} & \textbf{Soluzione Ottima} \\ \hline
  \textit{berlin52.tsp} & Berlino & 52 & 7542 \\ 
  \textit{burma14.tsp} & Birmania (Myanmar) & 14 & 3323 \\ 
  \textit{ch150.tsp} & Random & 150 & 6528 \\ 
  \textit{d493.tsp} & Foratura di circuiti stampati & 493 & 35002 \\ 
  \textit{dsj1000.tsp} & Random & 1000 & 18659688 \\ 
  \textit{eil51.tsp} & Sintetico & 51 & 426 \\ 
  \textit{gr202.tsp} & Europa & 202 & 40160 \\ 
  \textit{gr229.tsp} & Asia/Australia & 229 & 134602 \\ 
  \textit{kroA100.tsp} & Random & 100 & 21282 \\ 
  \textit{kroD100.tsp} & Random & 100 & 21294 \\ 
  \textit{pcb442.tsp} & Foratura di circuiti stampati & 442 & 50778 \\ 
  \textit{ulysses16.tsp} & Mediterraneo & 16 & 6859 \\ 
  \textit{ulysses22.tsp} & Mediterraneo & 22 & 7013 \\ \hline
  \end{tabular}
  \caption{Soluzioni ottime per ogni algoritmo su ogni dataset.}
  \label{tab:opt-results}
  \end{table}

\subsubsection{Risultati e risultati temporali}


I risultati sperimentali degli algoritmi da noi implementati sono riportati nella tabella alla pagina che segue.

\begin{landscape}
  %\thispagestyle{empty}
  % Please add the following required packages to your document preamble:
% \usepackage{multirow}
% Please add the following required packages to your document preamble:
% \usepackage{multirow}
\begin{table}[]
  \centering
  \begin{tabular}{|l|l|l|l|l|l|l|l|l|l|}
  \hline
  \multicolumn{1}{|c|}{\multirow{2}{*}{\textbf{Istanza}}} & \multicolumn{3}{c|}{\textbf{Held-Karp}} & \multicolumn{3}{c|}{\textbf{Nearest Neighbor}} & \multicolumn{3}{c|}{\textbf{2-approximation}} \\ \cline{2-10} 
  \multicolumn{1}{|c|}{} & \multicolumn{1}{c|}{\textbf{Soluzione}} & \multicolumn{1}{c|}{\textbf{\begin{tabular}[c]{@{}c@{}}Tempo\\ {[}s{]}\end{tabular}}} & \multicolumn{1}{c|}{\textbf{\begin{tabular}[c]{@{}c@{}}Errore\\ {[}\%{]}\end{tabular}}} & \multicolumn{1}{c|}{\textbf{Soluzione}} & \multicolumn{1}{c|}{\textbf{\begin{tabular}[c]{@{}c@{}}Tempo \\ {[}s{]}\end{tabular}}} & \multicolumn{1}{c|}{\textbf{\begin{tabular}[c]{@{}c@{}}Errore\\ {[}\%{]}\end{tabular}}} & \multicolumn{1}{c|}{\textbf{Soluzione}} & \multicolumn{1}{c|}{\textbf{\begin{tabular}[c]{@{}c@{}}Tempo\\ {[}s{]}\end{tabular}}} & \multicolumn{1}{c|}{\textbf{\begin{tabular}[c]{@{}c@{}}Errore\\ {[}\%{]}\end{tabular}}} \\ \hline
  \textit{berlin52.tsp} & 17739 & 180.0000379 & 135.20\% & 8980 & 0.0014364 & 19.07\% & 10114 & 0.0025047 & 34.10\% \\ 
  \textit{burma14.tsp} & 3323 & 0.3493254 & 0\% & 4048 & 0.0001614 & 21.82\% & 3814 & 0.0002581 & 14.78\% \\ 
  \textit{ch150.tsp} & 48029 & 180.0001351 & 635.74\% & 8191 & 0.0109101 & 25.47\% & 8347 & 0.0214446 & 27.86\% \\ 
  \textit{d493.tsp} & 111941 & 180.0005959 & 219.81\% & 41660 & 0.1401718 & 19.02\% & 44892 & 0.2299229 & 28.26\% \\ 
  \textit{dsj1000.tsp} & 551275688 & 180.0014467 & 2854.37\% & 24630960 & 0.5707007 & 32.00\% & 25086767 & 0.7079459 & 34.44\% \\ 
  \textit{eil51.tsp} & 1026 & 180.0000473 & 140.85\% & 511 & 0.0013854 & 19.95\% & 581 & 0.0020308 & 36.38\% \\ 
  \textit{gr202.tsp} & 55127 & 180.0001963 & 37.27\% & 49336 & 0.0205211 & 22.85\% & 51990 & 0.0374635 & 29.46\% \\ 
  \textit{gr229.tsp} & 176680 & 180.0002368 & 31.26\% & 162430 & 0.0284802 & 20.67\% & 180152 & 0.0480593 & 33.84\% \\ 
  \textit{kroA100.tsp} & 166257 & 180.0000981 & 681.21\% & 27807 & 0.004863 & 30.66\% & 27210 & 0.0076621 & 27.85\% \\ 
  \textit{kroD100.tsp} & 146862 & 180.0000945 & 589.69\% & 26947 & 0.0049283 & 26.55\% & 27112 & 0.009628 & 27.32\% \\ 
  \textit{pcb442.tsp} & 204852 & 180.0005189 & 303.43\% & 61979 & 0.1128042 & 22.06\% & 73030 & 0.1584171 & 43.82\% \\ 
  \textit{ulysses16.tsp} & 6859 & 1.9638338 & 0\% & 9988 & 0.0002 & 45.62\% & 7903 & 0.0003195 & 15.22\% \\ 
  \textit{ulysses22.tsp} & 7105 & 180.0000249 & 1.31\% & 10586 & 0.0003321 & 50.95\% & 8401 & 0.0005173 & 19.79\% \\ \hline
  \end{tabular}
  \caption{Risultati dell'esecuzione dei tre algoritmi sui dataset.}
  \label{tab:results}
  \end{table}
\end{landscape}

% ==================================================================================================

Riteniamo utile, inoltre, riportare nel dettaglio i tempi di esecuzione degli algoritmi e il numero di ripetizioni effettuate da ogni ogni algoritmo su ogni dataset.

\begin{table}[H]
  \centering
  \begin{tabular}{|l|l|r|r|r|r|r|r|}
  \hline
  \textbf{File} & \textbf{N} & \multicolumn{1}{l|}{\textbf{\begin{tabular}[c]{@{}l@{}}Tempo\\ \\ H-K {[}s{]}\end{tabular}}} & \multicolumn{1}{l|}{\textbf{\begin{tabular}[c]{@{}l@{}}Rep\\ HK\end{tabular}}} & \multicolumn{1}{l|}{\textbf{\begin{tabular}[c]{@{}l@{}}Tempo\\ NN {[}s{]}\end{tabular}}} & \multicolumn{1}{l|}{\textbf{\begin{tabular}[c]{@{}l@{}}Rep\\ NN\end{tabular}}} & \multicolumn{1}{l|}{\textbf{\begin{tabular}[c]{@{}l@{}}Tempo\\ 2-ap {[}s{]}\end{tabular}}} & \multicolumn{1}{l|}{\textbf{\begin{tabular}[c]{@{}l@{}}Rep\\ 2-ap\end{tabular}}} \\ \hline
  \textit{berlin52} & 52 & 180.0000379 & 1 & 0.0014364 & 663 & 0.0025047 & 385 \\
  \textit{burma14} & 14 & 0.3493254 & 2 & 0.0001614 & 5913 & 0.0002581 & 3705 \\ 
  \textit{ch150} & 150 & 180.0001351 & 1 & 0.0109101 & 91 & 0.0214446 & 46 \\ 
  \textit{d493} & 493 & 180.0005959 & 1 & 0.1401718 & 7 & 0.2299229 & 4 \\ 
  \textit{dsj1000} & 1000 & 180.0014467 & 1 & 0.5707007 & 1 & 0.7079459 & 1 \\ 
  \textit{eil51} & 51 & 180.0000473 & 1 & 0.0013854 & 701 & 0.0020308 & 482 \\ 
  \textit{gr202} & 202 & 180.0001963 & 1 & 0.0205211 & 48 & 0.0374635 & 26 \\ 
  \textit{gr229} & 229 & 180.0002368 & 1 & 0.0284802 & 35 & 0.0480593 & 20 \\ 
  \textit{kroA100} & 100 & 180.0000981 & 1 & 0.004863 & 203 & 0.0076621 & 129 \\ 
  \textit{kroD100} & 100 & 180.0000945 & 1 & 0.0049283 & 201 & 0.009628 & 103 \\ 
  \textit{pcb442} & 442 & 180.0005189 & 1 & 0.1128042 & 8 & 0.1584171 & 6 \\ 
  \textit{ulysses16.tsp} & 16 & 1.9638338 & 1 & 0.0002 & 4706 & 0.0003195 & 3046 \\ 
  \textit{ulysses22.tsp} & 22 & 180.0000249 & 1 & 0.0003321 & 2832 & 0.0005173 & 1911 \\ \hline
  \end{tabular}
  \caption{Dettaglio dei tempi di esecuzione.}
  \label{tab:exec-times}
  \end{table}

\subsection{Domanda 2}
\textit{Commentate i risultati che avete ottenuto: come si comportano gli algoritmi rispetti alle varie istanze?
C'è un algoritmo che riesce sempre a fare meglio degli altri rispetto all'errore di approssimazione? Quale dei tre
algoritmi che avete implementato è più efficiente?}

Analizzando i tre algoritmi abbiamo appurato che non esiste una regola generale per quanto riguarda l'accuratezza del risultato. Il più preciso dei tre è, naturalmente, l'algoritmo Held-Karp, in quanto per definizione calcola la soluzione ottima; nonstante questo, avendo una complessità di gran lunga maggiore rispetto agli altri, nei tre minuti che abbiamo imposto come limite alla computazione su un singolo dataset molto spesso si trova molto più lontano dalla soluzione rispetto agli algoritmi di approssimazione. Questo è particolarmente evidente nei grafi con un elevato numero di vertici; in \textit{dsj1000}, infatti, la soluzione trovata ha un errore di addirittura due ordini di grandezza in più rispetto alla soluzione ottima. \\
Anche tra gli algoritmi di approssimazione non è possibile tracciare una linea precisa su quale sia più preciso o meno: dai risultati ottenuti possiamo evincere solamente che l'algoritmo di 2-approssimazione tende a essere più preciso su grafi di piccole dimensioni, mentre l'euristica nearest neighbor tende a trovare una soluzione più vicina all'ottimo per grafi di grandi dimensioni. 